%%%%%%%%%%%%%%%%%%%%%%%%%%%%%%%%%%%%%%%%%
% Plasmati Graduate CV
% LaTeX Template
% Version 1.0 (24/3/13)
%
% This template has been downloaded from:
% http://www.LaTeXTemplates.com
%
% Original author:
% Alessandro Plasmati (alessandro.plasmati@gmail.com)
%
% License:
% CC BY-NC-SA 3.0 (http://creativecommons.org/licenses/by-nc-sa/3.0/)
%
% Important note:
% This template needs to be compiled with XeLaTeX.
% The main document font is called Fontin and can be downloaded for free
% from here: http://www.exljbris.com/fontin.html
%
%%%%%%%%%%%%%%%%%%%%%%%%%%%%%%%%%%%%%%%%%

%----------------------------------------------------------------------------------------
%	PACKAGES AND OTHER DOCUMENT CONFIGURATIONS
%----------------------------------------------------------------------------------------

\documentclass[a4paper,10pt]{article} % Default font size and paper size

\usepackage{fontspec} % For loading fonts
\defaultfontfeatures{Mapping=tex-text}
\setmainfont{Open Sans} % Main document font

\usepackage{xunicode,xltxtra,url,parskip} % Formatting packages

\usepackage[usenames,dvipsnames]{xcolor} % Required for specifying custom colors

\usepackage[big]{layaureo} % Margin formatting of the A4 page, an alternative to layaureo can be \usepackage{fullpage}
% To reduce the height of the top margin uncomment: \addtolength{\voffset}{-1.3cm}
%\usepackage{geometry}
%\geometry{lmargin=1in }

\usepackage{hyperref} % Required for adding links	and customizing them
\definecolor{linkcolour}{rgb}{0,0.2,0.6} % Link color
\hypersetup{colorlinks,breaklinks,urlcolor=linkcolour,linkcolor=linkcolour} % Set link colors throughout the document

\usepackage{titlesec} % Used to customize the \section command
\titleformat{\section}{\Large\raggedright}{}{0em}{}[\titlerule] % Text formatting of sections
\titlespacing{\section}{0pt}{3pt}{3pt} % Spacing around sections
\titleformat{\subsection}{\large\raggedright}{}{0em}{} % Text formatting of sections

\voffset=-10pt
\textheight=730pt

\begin{document}

\pagestyle{empty} % Removes page numbering

\font\fb=''[cmr10]'' % Change the font of the \LaTeX command under the skills section

%----------------------------------------------------------------------------------------
%	NAME AND CONTACT INFORMATION
%----------------------------------------------------------------------------------------

\par{\centering{\Huge Peter Borenstein}\bigskip\par} % Your name
\noindent\rule{14.3cm}{0.4pt}

\begin{tabular}{rl}
Address: &11247 San Jose Blvd Apt 1218, Jacksonville, Florida 32223 \\
Phone: & 904-703-1089\\
Email: & \href{mailto:pcborenstein@gmail.com}{pcborenstein@gmail.com}
\end{tabular}

%----------------------------------------------------------------------------------------
%	WORK EXPERIENCE
%----------------------------------------------------------------------------------------

\section{Work Experience}

\begin{tabular}{p{1.6cm}|p{12.4cm}}
%\begin{tabular}{r|p{11cm}}


\centering Jan 2016 &  Firmware Engineer at Critical Alert Systems, Jacksonville, FL \\
\centering - & \emph{Designed, developed, and coded hospital signaling equipment}\\
\centering May 2018 & \footnotesize{Worked on a system which conveys calls triggered by patients to nurses. I created a new RS485 router, a screen device to display calls, replaced chime chips with an audio circuit for new alarms, and created firmware for vandal proof room devices.
For an Ethernet device close to release, I made minor firmware changes and created a Python script to place MAC addresses into the binary programming file and print a label.}\\
&\footnotesize{Used Texas Instrument's Real Time Operating System (TI-RTOS) to create a screen display of calls.}\\
&\footnotesize{All firmware was written in C. Microchip's MPLAB-X IDE (Netbeans based) was used for PIC architecture microcontrollers, and Code Composer Studio (Eclipse based) was used to program and debug Texas Instruments' ARM chips.}\\
&\footnotesize{Release, test, and debug code was versioned and saved with Git.}\\
&\footnotesize{Organized and pre-tested equipment for UL 1069 certification. Testing included 8kV ESD zaps, impact testing, flame testing, 100,000 button presses, etc.}\\
&\footnotesize{Schematic design, PCB layout, and 3D modeling in Mentor Graphic's PADs}\\
&\footnotesize{Manually performed audio testing on two designs with a foam padded box and sound dB meter.}\\
&\footnotesize{Tested a new power supply design at max load with dummy resistors to verify safe temperature rise.}\\
&\footnotesize{Tested device to device communication with Ethernet and RS485 circuits at max cable length.}\\
&\footnotesize{Worked with Advanced Circuits for rapid prototyping; QMS and Solutions for production}\\
\multicolumn{2}{c}{} \\

\centering Aug 2013 &  Co-Founder and Hardware Lead at Verigo, Gainesville, FL \\
\centering -& \emph{Designed, developed, debugged, and coded a low power wireless data logger }\\
\centering Jan 2016&\footnotesize{The devices use Bluetooth Low Energy to communicate logged temperature and humidity data to smartphones \& tablets. Passed FCC Part 15, ICC, and CE certifications.
Technical lead for two part time engineers brought on after our seed funding round. The two engineers performed PCB layout, python test scripts, and manual testing.}\\
&\footnotesize{Raspberry Pi, Python scripts, and the BlueZ Bluetooth stack were used to create a Linux PC automated test equipment for manufacturing. The automated end of line tester programmed flash memory, verified power consumption, verified the radio with a connection, took a temperature sample, sect calibration values and date, and recorded the programmed MAC address to the cloud.}\\
&\footnotesize{I used Git to save, track, and control versions of the C and Python source code.}\\
&\footnotesize{All firmware written with C in IAR's Embedded Workbench IDE. The core architecture was 8051.}\\
&\footnotesize{Schematic design, PCB layout, and 3D modeling in Cadsoft's Eagle.}\\
&\footnotesize{Created and performed automated environmental testing to verify humidity and temperature ranges.}\\
&\footnotesize{Manually tuned pF capacitance on quartz crystal PCB traces to verify correct radio frequency and time keeping. This process must be performed with every new PCB manufacture due to slight changes in the PCB stackup affecting trace capacitance.}\\
&\footnotesize{Created python test script to simulate dropped Bluetooth connections in a Linux environment on a Raspberry Pi.
The scripts replicated connnection problems and various discovered bugs (bogus settings, unexpected events, timer overflow, etc). The automated scripts were incorporated in a test suite applied to all firmware changes.}\\
&\footnotesize{Created test firmware in C to simulate full logs of dummy data to test memory. Test firmware accelerated radio activity to test battery life.}\\
&\footnotesize{Created test firmware used to create radio output patterns required for FCC part 15 certification.}\\
&\footnotesize{Worked with Advanced Circuits for rapid prototyping; Arrow and EMI for production}\\
\multicolumn{2}{c}{} \\

%------------------------------------------------

\centering Dec 2011 & Test Technology Intern at Intel, Fort Collins, CO \emph{}\\
\centering - & \emph{Simulated pre-Silicon Verilog}\\
\centering May 2011&\footnotesize{Created hundreds of test patterns for Broadwell server chips pre-Silicon. Modeled scan chains allowing others to run tests. Used Perl to automatically read though lengthy test logs for relevant data.}\\
&\footnotesize{Used SVN for version control of test materials.}\\
\end{tabular}

\pagebreak
%----------------------------------------------------------------------------------------
%	EDUCATION
%----------------------------------------------------------------------------------------

\section{Education}

\begin{tabular}{rl}
December 2013 & Bachelor of Science in Electrical Engineering, \textbf{University of Florida}\\
& \small{Major: Electrical Engineering  | \emph{Magna Cum Laude}  |  Digital Design Specialization}\\
&\normalsize GPA: 3.66/4.0\\
\end{tabular}
\subsection{Notable Projects}

\small
In Digital Computer Architecture, I used VHDL and an FPGA to create a pipelined MIPS processor which could perform 25 instructions.\\

In Electrical Engineering design 1, I used C and Altium to create a PCB for and programmed a digital barometer with an LCD output.\\

In Electrical Engineering design 2, I used C and Altium to create a wireless Bluetooth thermometer and a Bluetooth to WiFi bridge to host data for a PC browser to view.\\

In Digital Design, I used VHDL and an FPGA to design various small digital projects (FIR filter, VGA timing outputs, minimalist processor design, etc)\\

In Microprocessor Applications, I used assembly, C, and Altium's Atmel Studio IDE to demonstrate use of various peripherals on an ATmega chip (ADC sample, serial communication to PC terminal, GPIO keypad input, etc)\\

I attended a week long embedded software training from the Barr Group. The course was taught with lectures and programming exercises on an ARM Cortex development board, STMicroelectronic's STM32F Discovery. Attendees used the C programming language to create portable device drivers, use real-time operating systems, and and complete nearly a dozen hands-on programming exercises. I was the only person to finish the final project in which attendees were tasked with using Micrium's ROTS to make a dive watch. Button and potentiometer inputs simulated ascent, decent, and air tank filling. Time, depth, and safety warnings were output on an LCD.\\

%----------------------------------------------------------------------------------------
%	COMPUTER SKILLS
%----------------------------------------------------------------------------------------

\section{Skills}

\begin{tabular}{rp{12.4cm}}
Top Skills & C and Git\\
Basic Knowledge & Python, C++, Perl, Linux system programming, Bash, VHDL\\
IDEs Used & IAR Workbench, Code Composer Studio, Atmel Studio, MPLAB-X\\
OSs used & Windows Vista, 7, 8 and 10, Ubuntu, and Debian\\
PCB CAD Used & Eagle, PCB Artist, Altera, PADS\\
Tools Used &  Oscilloscopes for viewing digital (SPI, clock, etc) and analog (audio, switching power supply, etc) waveforms, Logic Analyzers, Digital Multimeters, SMU\\
Agile and Scrum & Used Jira and Team Foundation Server to optimize meaningful labor. I worked with a project manager to evaluate effort required for project tasks including development, test, and debugging. Priorities were evaluated according to required time and effort.\\
TI E2E MVP & I was recognized by Texas Instruments in 2014 for sharing expertise, solutions, and experience to help other community members.\\
\\
\end{tabular}
%----------------------------------------------------------------------------------------
%	SCHOLARSHIPS AND ADDITIONAL INFO
%----------------------------------------------------------------------------------------

\section{UF IEEE Student Chapter Involvement}

\begin{tabular}{rl}
2011-2012 Academic Year &  Secretary (elected) \\

2010-2011 Academic Year & Advertising Chair \\

2011 & 1\textsuperscript{st} place IEEE South East Conference Engineering Ethics Competition\\

Various Dates & Introduced local grade students to robotics with Lego Mindstorm\\
\end{tabular}


%----------------------------------------------------------------------------------------
\end{document}
